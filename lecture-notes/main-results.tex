\section{Transition of Databases}

We are now ready to introduce the framework constructed by \cite{main-result}.
This work states everything in terms of databases, so we first develop a key piece of terminology.
\begin{defn}[Database Property]
	A \emph{database property} on $\cD$ is a subset $\sfP\subseteq\cD$ of the set of databases $D$.
\end{defn}
We think of $\sfP$ as a predicate on a database $D\in\cD$.
That is, $D\in\sfP$ if and only if $D$ satisfies $\sfP$.
Moreover, by abuse of notation, we also use $\sfP$ as a projector on the space of databases.
As a projector, $\sfP$ leaves databases satisfying it unchanged, and maps databases not satisfying it to $0$.
$$
\sfP(D) = \begin{cases}
	D & \mbox{if } D\in\sfP \\
	0 & \mbox{if } D\notin\sfP
\end{cases}
$$

Now that we defined database properties, we can describe the evolution of a database by the \emph{transitions} between database properties.
First, suppose the cheating prover has $k$ parallel processors.
We then define two quantities.
The first one measures the likelihood that a database that satisfies $\mathsf{P}$
turns into a database that satisfies $\mathsf{P}'$ from one $k$-parallel query.
The second one measures the likelihood that a database that satisfies $\mathsf{P}$
turns into one that satisfies $\mathsf{P}'$ from a quantum algorithm performing $k$-parallel queries up to $q$ times.

\begin{defn}[Database transition capacity]
	Let $k\in\bbN, \sfP, \sfP'\subseteq\cD$. The \emph{transition capacity} is defined as
	$$\lBrack \sfP \xrightarrow{k} \sfP'\rBrack = \max_{x, y} \norm{\sfP' cO_{xy} \sfP}$$
	Furthermore, we define
	$$\lBrack\sfP\xRightarrow{k,q}\sfP']\rBrack:=\sup_{U_i}\norm{\sfP' cO^k U_{q-1} cO^k\ldots U_1 cO^k \sfP}.$$
\end{defn}

The two quantities we defined above are related by the following bound:
\begin{lem}
	\label{transition-multi-to-single}
	For any sequence of properties $\sfP_i$,
	$$\lBrack\neg \sfP_0\xRightarrow{k, q}\sfP_q]\rBrack
	\leq\sum_{i}\lBrack\neg \sfP_{i-1}\xrightarrow{k}\sfP_i\rBrack$$
\end{lem}

Before proving the bound, we first demonstrate it in action.
As an example, we want to show that it is unlikely to create a hash chain of length $q+1$ using only $q$ queries.
Take
$$\sfP_s=\mathsf{CHN}^{s+1}$$
as the collection of databases containing hash chains of length $s+1$,
then our goal is to bound
$$\lBrack\sfP_0\xRightarrow{k,q}\sfP_q\rBrack.$$
Using \Cref{transition-multi-to-single}, it is suffice to bound each of
$$\lBrack\neg \sfP_{i-1}\xrightarrow{k}\sfP_i\rBrack.$$
In other words, we have reduced the problem to showing that it is difficult to go from chain length $i$ to length $i+2$ in a single query.
We now prove \Cref{transition-multi-to-single}.

\begin{proof}
	We first expand the definition of the transition capacity and insert the term $\mathsf{P}_{q-1}+(I-\mathsf{P}_{q-1})$ which equals to the identity.
	The expression can then be simplified using basic properties of operator norm.
	\begin{align*}
		\lBrack\neg \mathsf{P}_0\xRightarrow{k,q}\mathsf{P}_q\rBrack
			&=\sup_{U_i}\norm{\mathsf{P}_q cO^k U_{q-1} cO^k\ldots U_1 cO^k (I-\mathsf{P}_0)} \\
			&=\sup_{U_i}\norm{\mathsf{P}_q cO^k (\mathsf{P}_{q-1}+(I-\mathsf{P}_{q-1})) U_{q-1} cO^k\ldots U_1 cO^k (I-\mathsf{P}_0)}\\
			&\leq\sup_{U_i}\norm{\mathsf{P}_q cO^k \mathsf{P}_{q-1} U_{q-1} cO^k\ldots U_1 cO^k (I-\mathsf{P}_0)}\\
			&\qquad+\sup_{U_i}\norm{\mathsf{P}_q cO^k (I-\mathsf{P}_{q-1}) U_{q-1} cO^k\ldots U_1 cO^k (I-\mathsf{P}_0)}\\
			&\leq\sup_{U_i} \norm{\mathsf{P}_{q-1} cO^k\ldots U_1 cO^k (I-\mathsf{P}_0)}+\sup_{U_i} \norm{\mathsf{P}_q cO^k (I-\mathsf{P}_{q-1})}\\
			&\leq\lBrack\neg \mathsf{P}_0\xRightarrow{k,q-1}\mathsf{P}_{q-1}\rBrack+[\![\neg \mathsf{P}_{q-1}\xrightarrow{k}\mathsf{P}_q]\!]
	\end{align*}
	The result follows from induction.
\end{proof}

Recall that in the example earlier with finding a $q+1$-chain,
we implicitly related an algorithm's success probability with the contents of its database.
We now fill in the mathematical details by generalizing a previous result from \cite{compressed-oracles}.
\begin{thm}
	\label{zhandry-database}
	Let $R\subseteq X^\ell\times Y^\ell$ be a relation.
	Let $\mathcal{A}$ be some algorithm that, with success probability $p$, outputs $(\vec{x}, \vec{y})\in R$ where $H(x)=y$.
	Then we have
	$$\sqrt{p}\leq\lBrack\bot\xRightarrow{k, q} \mathsf{P}^R\rBrack+\sqrt{\frac{\ell}{M}}$$
	where $\mathsf{P}^R=\set{D | D\cap R\ne\emptyset}$.
\end{thm}

Now we introduce ways to simplify and bound quantities of the form 
$\lBrack\neg\sfP_{i-1}\xrightarrow{k}\sfP_i\rBrack$.
To begin, we introduce some more terminology to help us describe database transitions.
\begin{defn}[Recognizability]
	A transition $\neg\mathsf{P}\rightarrow\mathsf{P}'$ is (strongly and uniformly) \emph{recognizable}
		by $\set{\mathsf{L}_i}$ if and only if $\mathsf{P}'\subseteq \cup_i\set{\mathsf{L}_i}\subseteq\mathsf{P}$.
\end{defn}

There are also other versions such as \emph{weak} or \emph{non-uniform} recognizabilities which we will unfortunately skip today.
For our current definition, it is easier to first consider the $\sfP=\sfP'$ case.
In that case,
$\mathsf{P}'\subseteq \cup_i\set{\mathsf{L}_i}\subseteq\mathsf{P}$
just gives us
$\mathsf{P}=\cup_i\set{\mathsf{L}_i}$,
which says that for any $D$ to satisfy $\mathsf{P}$, it is necessary and sufficient that $D$ satisfies one of the $\mathsf{L}_i$'s.
Particularly, $\set{P}$ always recognizes $\neg\sfP\rightarrow\sfP$,
even though this fact is not useful later.

We can bound transition capacities in terms of \emph{local} properties that recognize the transition.
Roughly speaking, a set of database properties $\set{\mathsf{L}_i}$ is local if the truth value of each $\mathsf{L}_i$ depends only on a constant number of $x_j$'s.

\begin{thm}
	\label{property-to-bound}
	Let $\mathsf{L}_i$ be local properties that strongly recognize $\neg\mathsf{P}\rightarrow\mathsf{P}'$, then we have
	$$[\![\mathsf{P}\xrightarrow{k}\mathsf{P}']\!]\leq\max_{x, D}\sqrt{10\sum_iP[U\in\mathsf{L}_i]}$$
	where $U$ is the uniform distribution.
\end{thm}

One highlight of this theorem is that the right-hand side is entirely classical.
In other words, this framework allows its users to analyze quantum adversaries using almost fully classical reasoning,
and the resulting analysis may even roughly follow that of a classical adversary.
Towards that goal, now we present a few formula to break down transition capacities.
First, we have a formula for removing some pathological event $\neg\mathsf{Q}$ such as hash collisions.
\begin{lem}
	$$
	\lBrack\sfP\xrightarrow{k}\sfP''\rBrack\leq
	\lBrack\sfP\xrightarrow{k} \sfP''\setminus Q\rBrack+
	\lBrack\sfP\xrightarrow{k} \sfP''\cap Q\rBrack.
	$$
\end{lem}
Next, we have union bounds.
\begin{lem}
For all properties $P, Q, \sfP''$,
$$
	\lBrack\sfP\xrightarrow{k}\sfQ_1\cup\sfQ_2\rBrack\leq
	\lBrack\sfP\xrightarrow{k}\sfQ_1\rBrack+
	\lBrack\sfP\xrightarrow{k}\sfQ_2\rBrack.
$$
\end{lem}
Moreover, we can sort the queries by splitting the domain $X=X_1\cup X_2$.
Here $\lBrack\sfP\xrightarrow{k} \sfP''|X\rBrack$ is defined by restricting the oracle's inputs to $x\in X$.
\begin{lem}
$$
	\lBrack\sfP\xrightarrow{k} \sfP''|X\rBrack\leq
	\lBrack\sfP\xrightarrow{k} P'|X'\rBrack+
	\lBrack\neg P'\xrightarrow{k} \sfP''|X''\rBrack.
$$
\end{lem}
This last formula is especially helpful to analyze protocols that use the construction
\begin{align*}
	H_1&=H(1, x) \\
	H_2&=H(0, x)
\end{align*}
to split a single hash function into two.
We will show its proof:
\begin{proof}
	Decompose
	$$cO_{x,y}=cO_{x',y'}cO_{x'',y''}$$
	so that $cO_{x',y'}$ acts on all $x\in X'$.
	Then we have
\begin{align*}
	\lBrack P\xrightarrow{k} P''|X\rBrack &=\max_{xy}\norm{P''cO_{xy}P}\\
			&=\max_{xy}\norm{P''cO_{x''y''}cO_{x'y'}P}\\
			&=\max_{xy}\norm{P''cO_{x''y''}(P' + (I - P'))cO_{x'y'}P}\\
			&\leq\max_{xy}\norm{P''cO_{x''y''}P' cO_{x'y'}P}
			+\max_{xy}\norm{P''cO_{x''y''}(I - P')cO_{x'y'}P}\\
			&\leq\max_{xy}\norm{P' cO_{x'y'}P}
			+\max_{xy}\norm{P''cO_{x''y''}(I - P')}\\
			&\leq
			\lBrack P\xrightarrow{k} P'|X'\rBrack +
			\lBrack \neg P'\xrightarrow{k} P''|X''\rBrack .
\end{align*}
\end{proof}

There are more formula for simplifying transition capacities which we do not have time to present.
With a simple enough transition capacity,
we might be able to identify local properties that recognize it.
Once we do, we can then apply \Cref{property-to-bound} to obtain a bound.
As an example, we will give a rough sketch of the analysis for proof of sequential work.
