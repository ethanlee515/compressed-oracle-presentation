\section{Motivations}

We will present the work \cite{main-result} which
\begin{enumerate}
	\item creates a new framework to reason about hash functions under the quantum setting, and
	\item uses said framework to show that an existing proof of sequential work protocol is secure against quantum adversaries.
\end{enumerate}
To motivate our discussion, we will start with a brief introduction of the more popular \emph{proof of work} scheme,
which is an important component of the bitcoin mining protocol.
A proof of work protocol is designed to be difficult to compute while easy to verify.

The common construction for proof of work uses hash functions as a main ingredient; for a review of hash functions, we refer the readers to the text \cite{intro-algo}.
The protocol is as follows:
\begin{enumerate}
	\item
		Let $H$ be some hash function.
		Let $n\in\bbN$ be the public \emph{difficulty} parameter.
	\item Verifier samples random $x\randsamp\zo{\lambda}$ and sends it to the prover.
	\item Prover computes some $y$, where $H(x, y)$ starts with $n$ zeroes.
	\item Prover sends $y$ to verifier who computes and checks $H(x, y)$.
\end{enumerate}

Intuitively, the protocol is hard for the prover because the only way to compute $y$ is by brute force,
due to the \emph{one-way} property of $H$.
On the other hand, it is easy for the verifier to check the prover's work simply by evaluating $H(x, y)$.
An additional observation is that it is simple for the prover to parallelize his computation,
as he can just try many different $y$'s at the same time.
This motivates our definition of the proof of \emph{sequential} work,
where we add an additionally want the protocol to be difficult to parallelize.
A proof of sequential work scheme is also called a \emph{time-locked puzzle},
since now there is no way for the prover to speed up the protocol.
This leads to its applications in blockchains.

As a starting point, an unparallelizable and difficult task is to construct a \emph{hash chain}.
That is, given an input $x$, compute $H^n(x)$ by applying $H$ for $n$ times starting on $x$.
This is unfortunately not a good protocol by itself for proof of sequential work, since it is difficult to verify.
However, we will use this idea in the actual construction later.
