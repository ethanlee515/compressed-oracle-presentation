\section{Motivations}

We will present the work \cite{main-result} which
\begin{enumerate}
	\item creates a new framework to reason about hash functions under the quantum setting, and
	\item uses said framework to analyze an existing proof of sequential work construction.
\end{enumerate}

To motivate our topic today, we start with a brief introduction of the more popular \emph{proof of work} scheme.
As some readers might have heard, proof of work is an important component of the bitcoin mining protocol.
A proof of work protocol is designed to be difficult to compute, yet easy to verify.
Its construction uses hash functions as a main ingredient; for a review of hash functions, we refer the readers to the text \cite{intro-algo}.
A common construction of the proof of work protocol is as follows:
\begin{enumerate}
	\item
		Let $H$ be some hash function.
		Let $n\in\bbN$ be some public \emph{difficulty} parameter.
	\item Verifier samples random $x\randsamp\zo{\lambda}$ and sends it to the prover.
	\item Prover computes some $y$ where $H(x, y)$ starts with $n$ zeroes.
	\item Prover sends $y$ to verifier who checks $H(x, y)$.
\end{enumerate}

Intuitively speaking, the protocol is hard for the prover because there is no other ways to compute $y$ except by brute force.
On the other hand, it is easy for the verifier to check the prover's work simply by evaluating $H(x, y)$.
An additional observation is that it is easy for the prover to parallelize his computation.
If the prover has access to parallel processors, he can try many different $y$'s at the same time.
This motivates our definition of the proof of \emph{sequential} work,
where we add an additional goal that we want the protocol to be difficult to parallelize.
A proof of sequential work scheme is also called a \emph{time-locked puzzle},
since now there is no way for the prover to speed up the protocol.
This fact leads to applications in blockchains.

A na\"ive way of creating a difficult task that cannot be parallelized involves constructing a \emph{hash chain}.
That is, given input $x$, compute $H^n(x)$ by applying the hash function $H$ on $x$ for $n$ times.
This is unfortunately not a good protocol for proof of sequential work, as it is difficult to verify.
However, we will see the idea of hash chains in the actual construction that we will see later.
